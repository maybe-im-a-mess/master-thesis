% !TeX encoding = UTF-8

%===============================================================================
% Font options are:
%   plain (default), serif (uses Palladio), sans-serif (uses Paratype Sans)
%
% Layout options are:
%   article (default, no chapters), book (for longer texts, offers \chapter)
%   Changing this value between LaTeX runs may require deleting the .aux files
%
% Paragraph options are:
%   noparskip (default, no spacing between paragraphs), parskip (spaced)
%
% Language options are:
%   de (default), en
%   Changing this value between LaTeX runs may require deleting the .aux files
%
\documentclass[serif,article,noparskip,de]{agse-thesis}

% Global parameters, replace with actual values.
\newcommand{\thesisTitle}{Über den Sinn des Lebens}
% -> You may use \par (but not \\) to format the title. If you do so, you'll
%    need to manually set the 'pdftitle' attribute below.
\newcommand{\studentName}{Hugo Schlupps}
%===============================================================================

\hypersetup{pdftitle={\thesisTitle}}
\hypersetup{pdfauthor={\studentName}}

\addbibresource{bib/bibliography.bib}

% Blind texts, for demonstration only, not part of the actual template
\usepackage{lipsum}

\begin{document}

\coverpage[
    student/id=1234567,
    student/mail=email@inf.fu-berlin.de,
    thesis/type=Bachelorarbeit,            % optional, default: Bachelorarbeit
    thesis/group={Arbeitsgruppe Software Engineering},
                                           % optional, default: AGSE
    thesis/advisor={Matt Visor},           % optional
    thesis/examiner={Prof. Dr. Mia Maus},
    thesis/examiner/2={Prof. Dr. Bob Bär}, % optional
    thesis/date=\today,                    % optional, default: \today
   %title/size=\LARGE,      % set this value to overwrite automatic font size
   %abstract/separate       % toggle this to move the abstract to its own page
]
{ % Abstract is stored in front/abstract.tex
    % !TeX encoding = UTF-8
% Short abstract for the thesis. Edit this file with the final abstract text.
\noindent
This thesis studies the meaning of life in the context of software engineering.
It provides an illustrative example of the AGSE thesis template reorganization.
% Replace the above lines with your actual abstract. Use \par for paragraph breaks.

}

\include{front/declaration}

\cleardoublepage

\tableofcontents

\cleardoublepage

\pagestyle{fancy}

% Actual content starts here

% !TeX encoding = UTF-8
\chapter{Introduction}

This chapter will introduce the topic of the thesis, motivation of the research and the problem statement. It will also outline the research objectives and provide an overview of the thesis structure.

\section{Motivation}

Driven by the rapid technological advancement, Large Language Models (LLMs) have integrated into a variety of applications, including virtual assistants, content generation, and chatbots among a wide range of domains, such as education, healthcare, or business. However, the deployment of LLMs in real-world tasks has revealed significant challenges, particularly in terms of their reliability and safety. 

\section{Problem statement}

TODO:
The challenges in ensuring LLMs extract correct information and communicate it effectively without hallucination or omitting critical details.

\section{Research Objectives}

TODO:
Outline the goals (benchmarking IE approaches, simulating patient-doctor interactions, and evaluating communication quality).

\section{Thesis outline}

TODO:
Briefly describe the content of each chapter, highlighting how they contribute to addressing the research objectives and solving the problem statement.
% !TeX encoding = UTF-8
\chapter{Fundamentals and related work}

This chapter will provide a review of fundamental topics and literature related to the main concepts used in this thesis, such as ...(TODO) The chapter will also provide an overview of the current state of research in the field of ... (TODO)

\section{Large Language Models in Healthcare}
TODO:
General overview of the application of LLMs in healthcare.

\section{Information Extraction}

TODO:
Discuss the theory behind Naive Prompting, Chain-of-Thought, Atomic Fact Extraction, and Unified Information Extraction.

\section{Citation Mechanisms}

TODO:
Discuss the theory behind citation mechanisms.

\section{Dialogue Management in Conversational AI}

TODO:
Theory behind state tracking and managing mandatory conversation turns.

\section{Evaluation Metrics for LLM systems}

TODO:
Overview of existing methods like LLM-as-a-Judge and traditional retrieval metrics.
\input{chapters/3_main}
\input{chapters/4_conclusion}

\printbibliography

\appendix
\include{chapters/5_appendix}

\end{document}
